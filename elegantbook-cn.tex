%!TEX program = xelatex
\documentclass[lang=cn,10pt]{elegantbook}

% ====================================
% 封面信息
% ====================================
\title{Soren's Literary Journey}
\subtitle{脆弱的文学作品集?}

\author{刘潼}
\date{\zhtoday}
\version{0.1}
\bioinfo{目前状态}{编辑中}

\extrainfo{何必为了虚无缥缈的可能,创造本不应该存在的期待,造成长久的失落 --- 刘潼}

\cover{cover2.jpg}

% ====================================
% 🎨 文学作品集专用样式定制
% ====================================

% 自定义主题色(淡雅蓝紫色)
\definecolor{customcolor}{RGB}{106, 141, 222}
\colorlet{coverlinecolor}{customcolor}

% --- 目录设置 ---
\setcounter{tocdepth}{2}         % 显示到 section(每篇作品)

% --- 重定义章标题格式(去掉「第X章」编号)---
\titleformat{\chapter}[display]
  {\bfseries\LARGE\centering}    % 格式:粗体、大号、居中
  {}                              % 标签:空(不显示「第X章」)
  {0pt}                           % 标签与标题间距
  {\color{structurecolor}}        % 标题样式
\titlespacing{\chapter}{0pt}{-20pt}{2\baselineskip}

% --- 重定义节标题格式(隐藏「1.1」编号)---
\titleformat{\section}[hang]
  {\large\bfseries}               % 格式
  {}                              % 标签:空
  {0pt}                           % 间距
  {\color{structurecolor}}        % 样式
\titlespacing{\section}{0pt}{1.5\baselineskip}{0.5\baselineskip}

% --- 目录样式定制 ---
\usepackage{titletoc}

% 章目录样式(分类大标题)
\titlecontents{chapter}
  [0pt]                                           % 左边距
  {\vspace{1.2em}\large\bfseries\color{structurecolor}}  % 格式
  {}                                               % 编号格式
  {}                                               % 无编号格式
  {\hfill\contentspage}                           % 页码

% 节目录样式(每篇作品)
\titlecontents{section}
  [1.5em]                                         % 左边距(缩进)
  {\normalsize}                                   % 格式
  {}                                               % 编号格式
  {}                                               % 无编号格式
  {\titlerule*[0.6pc]{.}\contentspage}            % 点线+页码

% --- 作品分隔符 ---
\newcommand{\worksep}{%
  \vspace{1.5em}
  \begin{center}
    \color{structurecolor!50}
    {\adforn{34}\quad\adforn{34}\quad\adforn{34}}
  \end{center}
  \vspace{1em}
}

% 诗歌环境已在 elegantbook.cls 中定义
% 可用命令: \begin{poem}, \stanzabreak, \poemsep, \poemsign{落款}


% ====================================
\begin{document}

\maketitle
\frontmatter

% ====================================
% 序
% ====================================
\chapter*{序}
\markboth{序}{序}

这是我的文字,是我与世界对话的方式。

不求惊世骇俗,只愿字字真诚。

\begin{flushright}
刘潼\\
\zhtoday
\end{flushright}

\tableofcontents

\mainmatter

% ====================================
% 随笔
% ====================================
\chapter{随笔}

\section{我为什么写文章}

我写文章,不是因为我有什么非说不可的话,而是因为我需要一种方式来整理那些杂乱的思绪。

有时候,我觉得写作就像是在黑暗中点燃一根火柴——它不能照亮整个世界,但至少可以让我看清脚下的路。

写作是孤独者的出口,是沉默者的声音。当我无法对任何人诉说时,我选择对着空白的纸张倾诉。纸张不会评判,文字不会背叛。

\worksep

\section{论孤独}

孤独并不等同于独处。

一个人可以在人群中感到深深的孤独,也可以在独处时感到无比的充实。孤独是一种心灵的状态,是自我与世界之间的一种微妙距离。

我曾经害怕孤独,后来我学会与它共处,再后来我发现——孤独是最好的老师。

它教会我倾听自己的声音,教会我在喧嚣中保持清醒,教会我珍惜那些真正懂得我的人。

\worksep

\section{时间的味道}

时间有味道吗?

童年的时光,尝起来像是夏日午后的冰棍,甜腻而短暂。

青春的岁月,带着青涩的苹果味,酸涩中透着希望。

而现在,时间的味道变得复杂,像是一杯陈年的酒,需要慢慢品味。

\worksep

\section{城市的夜}

城市的夜晚从来不是真正的黑暗。

霓虹灯将天空染成暧昧的紫红色,车流在街道上穿行,像血液在城市的血管中流动。

街角的便利店,24小时不灭的灯光;写字楼里点亮的几扇窗户;地铁站最后一班列车的轰鸣;清晨四点面包店飘出的香气。

这是城市的呼吸,永不停息。

而我,只是这座城市里的一粒尘埃,随风漂泊,无处安放。

% ====================================
% 诗
% ====================================
\chapter{诗}

\section{夜行者}

\begin{poem}
星光落在肩上\\
像一封未读的信\\
我在城市的伤口中行走\\
脚步声敲打着沉默的街道

\stanzabreak

没有人知道我来自何方\\
也没有人关心我将去向何处\\
我只是夜的一部分\\
流动的,模糊的,匿名的
\end{poem}
\poemsign{2024年深秋}

\poemsep

\section{窗}

\begin{poem}
透过这扇窗\\
我看见另一个世界\\
那里有人欢笑\\
那里有人哭泣

\stanzabreak

而我站在这边\\
像一个旁观者\\
像一滴落不下的泪\\
悬在时光的边缘
\end{poem}
\poemsign{写于某个雨夜}

\poemsep

\section{秋思}


\begin{poem}
西风萧萧叶纷飞,\\
孤雁南归人未归。\\
月照空庭霜满院,\\
独倚栏杆望天涯。
\end{poem}
\poemsign{仿古风}

\poemsep

\section{无题}

\begin{poem}
咖啡凉了\\
思念却刚刚加热\\
你不在的日子\\
时间格外缓慢
\end{poem}


% ====================================
% 故事
% ====================================
\chapter{故事}

\section{最后一班地铁}

末班地铁的车厢里只有三个人。

一个疲惫的上班族,靠着窗户假寐,领带还打得一丝不苟,但眼角的皱纹出卖了他的年龄。

一个戴着耳机的年轻女孩,盯着手机屏幕,偶尔露出若有若无的笑容——也许,她正在和远方的恋人发消息。

还有我,一个刚刚结束加班的程序员,背包里装着一台沉甸甸的笔记本电脑和今天所有未完成的焦虑。

地铁穿过一个又一个站台,每一站都有人下车,有人上车。我们像是这个城市的细胞,在它的血管中来来往往,彼此擦肩而过,却从不真正相遇。

窗外是漆黑的隧道,偶尔闪过的灯光像是时间的刻度,提醒着我们正在穿越城市的腹地。

那个假寐的男人突然醒了,揉了揉眼睛,看了一眼站牌,又闭上了眼。

女孩收起手机,望着窗外的黑暗发呆。

而我,想起了很久以前,也有一个深夜,我坐在同样的地铁上,旁边坐着一个人。

那个人,后来去了很远的地方。

列车停靠,到站了。

我们三个人,走向各自的方向,消失在各自的夜色中。

明天,又是新的一天。

% ====================================
% 读书札记
% ====================================
\chapter{读书札记}

\section{《边城》}

沈从文笔下的湘西世界,是现代文明尚未侵蚀的净土。

那里的山水、人情、风俗,构成了一幅中国式田园牧歌的画卷。翠翠的爱情故事,简单而又复杂。

它简单,因为那是一种纯粹的、未经世俗污染的情感。

它复杂,因为命运的捉弄让这份感情最终没有结果。

读完这本书,我久久无法平静。也许,最美的东西总是最脆弱的。就像翠翠的等待,就像那条河,就像那个已经回不去的年代。

\worksep

\section{《百年孤独》}

``多年以后,面对行刑队,奥雷里亚诺·布恩迪亚上校将会回想起父亲带他去见识冰块的那个遥远的下午。''

这是我读过的最震撼的开头之一。

马尔克斯用一种魔幻的方式,书写了一个家族七代人的命运。在这个故事里,时间是循环的,命运是注定的,孤独是永恒的。

每个人都在试图打破孤独,却都在孤独中走向毁灭。

也许,这就是人类的宿命。

\end{document}
